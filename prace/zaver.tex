\chapter*{Závěr}
\addcontentsline{toc}{chapter}{Závěr}

\noindent
Výsledný produkt se vyvinul nad očekávání autora. Rozpoznávání SPZ 
pomocí běžného telefonu funguje skvěle a
občas i za poměrně nehostinných podmínek (viz obrázek \ref{fig:recogn}).

\begin{figure}[!htb] \centering
  \includegraphics[width=145mm]{../img/example_recognition.png}
  \caption{Rozpoznaná SPZ v nehostinných podmínkách.}
  \label{fig:recogn}
\end{figure}

\noindent
Co se týče uživatelského rozhraní, tak to vypadá jednotně a přehledně.
Je nabité funkcemi a dohromady s backendem umožňuje ovládat parkovací systém. Zejména monitorovat
zařízení a měnit jejich nastavení na dálku, simulovat vytvořená parkovací pravidla
a filtry, číst statistiky a záznamy parkování.

Vývoj byl díky vhodně zvoleným technologiím poměrně rychlý a bezbolestný.
V jeho průběhu nedošlo k žádnému backtrackování kvůli předchozím rozhodnutím.
I automatické testování se vyplatilo.
Projekt je bez velkých obtíží rozšiřitelný a pozměnitelný.
Projekt by šlo rozšířit o komunikaci se závorou a platebním terminálem,
což by z projektu udělalo plnohodnotný parkovací systém.

\section*{Splnění zadání}

\noindent
Výsledný produkt splňuje zadání (viz \ref{zadani}), kde nejtěžší funkcionalitou
byla funkcionalita popsaná v bodě 2b, který byl splněn možností kopírování pravidel
na libovolná data na stránce s pravidly. Nejtěžší byla, protože vyžadovala vytvoření
dalšího měsíčního pohledu, aby tato funkcionalita byla lehce použitelná,
a logika uživatelského rozhraní je složitá.

% Výsledný projekt splňuje celé zádání kromě jednoho bodu -- konkrétně se jedná o
% různý provoz
% o svátích a podobných dnech. Implementace takovéto funkcionality
% vyžaduje implementovat kalendář.
% Co se týče rozpoznávání SPZ, statistik, samotných pravidel a filtrů vozidel, je zde
% vše implementováno a funguje skvěle.
% Při prohlížení záznamů parkování lze nahlédnout na výřez SPZ z pořízeného snímku.
% Navíc lze pravidla a filtry simulovat pro libovolné vozidlo.

% Vývoj byl díky vhodně zvoleným technologiím poměrně rychlý a bezbolestný.
% V jeho průběhu nedošlo k žádnému backtrackování kvůli předchozím rozhodnutím.
% Projekt je bez velkých obtíží rozšiřitelný a pozměnitelný.
% Dalším rozšířením, aby projekt byl plnohodnotný parkovací systém, by byla
% integrace s platebním terminálem a závorou.
