\chapter{Úvod} \label{uvod}

\noindent
Tento dokument se zabývá koncepcí a implementací informačního systému pro uzavřená
parkoviště s automatickým rozpoznáváním SPZ bez použití drahého kamerového hardware.
Inspirací pro projekt bylo, že se po autorovi při výjezdu z parkoviště nechtěl
parkovací lístek, nýbrž vozidlo bylo puštěno na základě SPZ.

Projekt se nezabývá operací se závorou a platebním terminálem, neb by to zvyšovalo
obtížnost už tak obtížného projektu a vyžadovalo by to poměrně vysoké
náklady na hardware.

Projekt zároveň slouží k vyzkoušení si moderních webových technologií
a frameworků s přesahem k mobilnímu vývoji a počítačovému vidění.

% Bude se tedy vytvářet parkovací systém, který má následující
\section*{Cíle}

\begin{itemize}
  \setlength\itemsep{0.05em}
  \item Rozpoznávání SPZ levným hardwarem.
  \item Vytváření flexibilních parkovacích pravidel.
  \item Přívětivé uživatelské rozhraní.
  \item Statistiky a pěkné grafy.
\end{itemize}

% Parkovací systém by měl mít následující funkcionalitu:

% \begin{itemize}
%   \setlength\itemsep{0.05em}
%   \item Naskenování SPZ.
%   \item Vytváření dostatečně flexibilních pravidel pro většinu použití.
%   \begin{itemize}
%     \setlength\itemsep{0.05em}
%       \item Od času A do B za tarif X/jednotka času.
%       %\item Různý provoz o svátcích a víkendech. \label{missing1}
%       \item Limity na hodiny zdarma.
%       \item Filtrování vozidel -- pro různá vozidla mohou platit jiná pravidla.
%     \end{itemize}
%   \item Statistiky počtu aut a výdělku s grafy.
% \end{itemize}
