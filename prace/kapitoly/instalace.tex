
\chapter{Instalace} \label{analyza}

Instalace celého systému je velice jednoduchá. Pro distribuci je použit nástroj
Docker \citep[viz][]{DockerDocs}, který zajistí kompilaci, spuštění, propojení všech částí a odhalení
potřebných služeb ven na internet.

Kromě Backendu, Frontendu, OpenALPR Server a MongoDB používá tato distribuce
Dockerem populární HTTP server NGINX pro SSL a reverse proxy. Vnitřní HTTP komunikace
tak nemusí být zabezpečena.

\subsubsection*{Postup Instalace}

\begin{enumerate}
  \item Po stažení git repozitáře se zdrojovým kódem nastavíme submoduly:\\
  \begin{lstlisting}
    $ git submodule init
    $ git submodule update --remote --recursive
  \end{lstlisting}
  \item Dle potřeby upravíme soubor \textit{/docker-compose.yml} kvůli hostname, SSL certifikátům pro NGINX apod.
  \item Celý systém spustíme.\\
  \begin{lstlisting}
    $ docker-compose up
  \end{lstlisting}
\end{enumerate}

Poslední krok může trvat i několik minut v závislosti na rychlosti internetového
přípojení. Protože Docker zabalí vše včetně systémových závislostí, velikost
výsledných imagů je $1,4$GB.
