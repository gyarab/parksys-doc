
\chapter{Webová aplikace}

\section{Obrazovky}

Dle požadavků z úvodu mějme po příhlášení do webového rozhraní následující obrazovky, mezi kterými bude uživatel přepínat pomocí hlavního menu.

\begin{itemize}
  \item \textbf{Dashboard} -- jednoduché shrnutí nedávných statistik, kolik vozidel je
        momentálně na parkovišti, aktivní zařízení apod.
  \item \textbf{Statistiky} -- detailněji zobrazené údaje o počtu parkování a výdělku podle roku, měsíce a dne s grafy.
        Zde půjde i exportovat data to CSV souboru (\textit{Comma-Separated-Values}).
  \item \textbf{Pravidla a Filtry} -- definice parkovacích pravidel a filtrů vozidel (popsáno v \ref{analysis_parking_schema}).
        Pro ověření půjde si parkovací pravidla a filtry odsimulovat.
  \item \textbf{Zařízení} -- správa zařízení zachycujících fotografie SPZ, která lze autentifikovat do systému pomocí
        QR kódu.
  \item \textbf{Správa uživatelů} -- přidávání, odebírání a úprava uživatelů a jejich operávnění.
  \item \textbf{Správa účtu} -- změna údajů a hesla současně přihlášeného uživatele.
\end{itemize}

\section{Autentizace a autorizace} \label{auther_authen}

Autentifikace lidských uživatelů bude probíhat pomocí standardního hesla a zařízení pomocí dlouhého, náhodně generovaného hesla,
které bude posíláno na Frontend ve formě QR kódu, které zařízení naskenuje.
TODO: Tokens

Autorizace pak bude probíhat bez Cookies, místo toho bude Access Token poslán v hlavičce HTTP hlavičce
\textit{Authorization}, čímž předejdeme CSRF/XSRF útokům.

\section{Business logika}

\subsection{Parkovací pravidla} \label{analysis_parking_schema}

\begin{itemize}
  \item Různá vozidla mohou podléhat ruzným pravidlům.
  \item Pravidla mají prioritu.
  \item Pravidla mají časové omezení.
  \item V jednu chvíli může platit více pravidel.
\end{itemize}

Mechanismus, kterým umožníme vozidlům být ovlivněna některými pravidly,
budou filtry.
Pro dostatečnou flexibilitu je zapotřebí oddělit samotná pravidla od jejich
priority, časového intervalu platnosti i filtrů,
k čemuž bude sloužit objekt typu \texttt{ParkingRuleAssignment}.

% TODO - add a db schema image that takes less space
\begin{lstlisting}
type ParkingRuleAssignment {
  rules: [ParkingRule]!
  start: DateTime!
  end: DateTime!
  # ALL nebo NONE
  vehicleFilterMode: VehicleFilterMode!
  vehicleFilters: [VehicleFilter!]
  priority: NonNegativeInt!
}

type VehicleFilter {
  id: ID!
  name: String!
  action: VehicleFilterAction!
  vehicles: [Vehicle!]!
}
\end{lstlisting}

Filtrování bude mít dva módy: začneme se všemi vozidly (ALL) a začneme bez vozidel (NONE).
Následné filtry mohou buď přidávat, nebo odstraňovat jednotlivá vozidla.
Hodí se mít filtry uložené separátně, aby mohli být využity několikrát.

Pro zjednodušení algoritmů, uvalíme omezení: ve stejný čas nesmí existovat více \texttt{ParkingRuleAssignment}
se stejnou prioritou.

\subsubsection*{Algoritmus filtru vozidel}

\textit{Vstup: objekt \texttt{ParkingRuleAssignment}, vozidlo}

\textit{Výstup: boolean určující platnost}
\begin{enumerate}
  \item Na základě módu filtrování si budeme udržovat množinu buď odstraněných vozidel (mód ALL), nebo přidaných vozidel (mód NONE).
  \item Podle příslušné akce filtrů (odstranit nebo přidat) budeme množinu našich vozidel manipulovat. Např. je-li mód ALL a filtr odstraňuje, do množiny si vozidla přidáme.
  \item Pokud je vozidlo ve výsledné množině, tak pro něj \texttt{ParkingRuleAssignment} platí pokud je filtrovací mód NONE a neplatí pokud je mód ALL. Opačné výsledky nastanou, pokud vozidlo v množině není.
\end{enumerate}

Časová i paměťová složitost algoritmu je ${\cal O}(N)$, kde $N$ je počet filtrů.

Tento algoritmus lze potenciálně rozdělit na předvýpočet (kroky 1. a 2.) a ověření (krok 3.).
To je jedna možnost, která zvýší paměťovou náročnost, protože bychom si museli pamatovat množiny, což je nepraktické.
Je-li uživatel příčetný, počet použitých filtrů nebude obrovský, a tudíž je lepší zvolit následující způsob cachovaní.
Zapamatujeme si výsledky pro určitý seznam filtrů pro konkrétní vozidlo pouze při běhu algoritmu, který je
vysvětlen v následující kapitole, čímž následující algoritmus zrychlíme.

\subsubsection*{Algoritmus pro aplikaci ParkingRuleAssignmentů}

\begin{figure}[!htb] \centering
  \includegraphics[width=145mm]{../img/rules_drawing.jpg}
  \textit{Modré úsečky neplatí kvůli filtrům nebo protože jsou deaktivované. Oranžová čára značí výstup požadovaného algoritmu.}
  \caption{Ilustrace problému úseček.}
  \label{fig:rules_drawing}
\end{figure}

V jednom čase může existovat více objektů typu \texttt{ParkingRuleAssignment} avšak s různou prioritou.
Může se stát, že aplikovaných \texttt{ParkingRuleAssignment} bude několik (různé priority, vyprší platnost, etc.).

Situaci si lze představit jako několik úseček navzájem rovnoběžných úseček v různých výškách, které se neprotínají.
Nás nyní zajímá, na které a v jakých intervalech na ně dopadne světlo, pokud na ně kolmo zeshora posvítíme.
Situaci lze vidět na obrázku \ref{fig:rules_drawing}.

Pro zjednodušení předpokládejme, že všechny \texttt{ParkingRuleAssignment}, které zpracováváme, platí pro naše vozidlo.
Přidat tuto kontrolu později je triviální.

\textit{Vstup: seznam \texttt{ParkingRuleAssignmentů} odpovídající pro interval pobytu vozidla na parkovišti, vozidlo}

\textit{Výstup: seznam \texttt{ParkingRuleAssignmentů} s časy platnosti}
\begin{enumerate}
  \item Seřadíme si začátky a konce úseček podle jejich času.
  \item Vytvoříme si haldu pro odkládání úseček, která řadí podle priority -- větší výše.
  \item Vytvoříme si seznam aplikovaných pravidel s časy (časy mohou se lišit od počátečních i koncových časů).
  \item Nechť \textit{s} je současná úsečka a \textit{$t_s$} čas zvolení \textit{s} (čas zvolení se může lišit od začátku úsečky).
  \item Pro každou událost \textit{u} značící začátek/konec úsečky (aplikaci pravidla) \textit{p}:
  \begin{enumerate}
    \item Pokud se jedná o začátek nové úsečky:
    \begin{enumerate}
      \item Pokud není zvolená úsečka:\\
            \textit{$t_s$} $\leftarrow$ \textit{p.start}\\
            \textit{s} $\leftarrow$ \textit{p}
      \item Pokud je zvolená úsečka a \textit{p} má vyšší prioritu než \textit{s}:\\
            \textit{s} dáme do seznamu aplikovaných pravidel se začátkem \textit{$t_s$} a koncem \textit{p.start}.\\
            \textit{s} dáme na haldu, pokud \textit{s.end} > \textit{p.end}.\\
            \textit{$t_s$} $\leftarrow$ \textit{p.start}\\
            \textit{s} $\leftarrow$ \textit{p}
      \item Pokud je zvolená úsečka a \textit{p} má nižší prioritu než \textit{s} a \textit{p.end} > \textit{s.end}:\\
            \textit{s} dáme na haldu
    \end{enumerate}
    \item Jinak (jedná se o konec nějaké úsečky):
    \begin{enumerate}
      \item Přidáme \textit{s} do seznamu aplikovaných pravidel se začátkem \textit{$t_s$} a koncem \textit{s.end}.
      \item Taháme z haldy, dokud nedostaneme úsečku s koncem později než konecm \textit{p}, nebo dokud halda není prázdná.
      \item Pokud jsme z haldy vhodnou úsečku vytáhli, použijeme ji. V opačném případě vyprázdníme \textit{s} a \textit{$t_s$}.
    \end{enumerate}
  \end{enumerate}
\end{enumerate}

Algoritmus zajisté doběhne, protože máme konečný počet událostí a v každém cyklu jednu zpracujeme.
Při rozumném počtu \texttt{ParkingRuleAssignment} v daném intervalu je algoritmus velice rychlý.
${\cal O}(N^2\cdot logN)$
${\cal O}(N\cdot logN)$
Přidáme-li filtrování, které nemusíme provést pro všechny úsečky, ${\cal O}(N\cdot M)$.

\section{GraphQL resolvery}

<TODO: EXPLAIN RESOLVERS>

Jelikož je většina operací nad modeli se opakuje, byli napsány obecné resolvery jako HOF (Higher-Order-Function)
pro vytváření, úpravu, mazání, vyhledávání a získávání modelů v relaci.
Měnící se část je pak pouze definice databázového modelu.

\section{Databázový model}

\section{Uživatelské rozhraní}

\subsection{Obecná vybírátka modelů}

Dle principu neopakování se bylo vytvořeno několik obecných UI komponent, které umožňují vyhledávání libovolných
modelů a jejich volbu a využití v ostatních komponentách. Ve zdrojovém kódu je implementujeme jako
HOF (Higher-Order-Function), což je funkce vracející další funkce -- konkrétní komponenty.
Měnící se části, které do těchto obecných komponent budeme vkládat je komponenta renderující jediný model
(\textit{renderModel}),
GraphQL dotaz pro získávání modelů (\textit{queryString}), funkce, jejímž vstupem je odpověď na GraphQL dotaz a výstupem je
samotné pole modelů (\textit{modelArrayGetter}), a funkce, jejímž vstupem je dotazovací řetězec a výstupem jsou
argumenty pro GraphQL dotaz (\textit{identifierToOptions}).

Obrázek \ref{fig:picker_lifecycle} ukazuje tok dat v jednom z obecných vybírátek.
Obrázek \ref{fig:picker_component} ukazuje vyrenderovanou komponentu téhož vybírátka.

\begin{figure}[!htb] \centering
  \includegraphics[width=145mm]{../img/picker_lifecycle.jpg}
  \caption{Tok dat v jednom z obecných vybírátek.}
  \label{fig:picker_lifecycle}
\end{figure}

\begin{figure}[!htb] \centering
  \includegraphics[width=70mm]{../img/picker_component.png}
  \caption{Vyrenderovaná komponenta jednoho vybírátka.}
  \label{fig:picker_component}
\end{figure}
