
\chapter{Rozpoznávání SPZ}

Jak již bylo řečeno v kapitole \ref{archtech}, mobilní aplikace pořídí snímek,
pošle ho na Backend, jenž ho pošle serveru s knihovnou OpenALPR, která
rozposná SPZ a výsledek pošle zpět na Backend. V této kapitole si popíšeme
mobilní aplikaci a server s OpenALPR.

\section{Volba zařízení pro mobilní aplikaci}

Co se týče hardwarového vybavení snímacího zařízení, tak je vyžadována přední kamera
s rozlišením alespoň 1000 na 1000 pixelů. Důvod pro toto rozlišení je, že Backend obdržený snímek
stejně zmenší na 1000x1000 pixelů, aby knihovna OpenALPR provedla rozpoznání co nejrychleji, a zároveň
aby bylo rozpoznání dostatečně přesné. Procesor, RAM i vnitřní paměť může být libovolná -- kterékoliv
dnešní nové zařízení bohatě postačí (za předpokladu, že vnitřní paměť není zaplněná).
Minimální verze Androidu je 5 (SDK 21).

Autorovi se nepodařilo najít způsob, jak zároveň pořizovat v pravidelném intervalu snímky
a mít zařízení uzamknuté proti přístupu. K zajištění pořizování snímků si hlavní
obrazovka aplikace řekně systému Android o \textit{FLAG\_KEEP\_SCREEN\_ON}, což zabrání uzamknutí.
To má dva následky. První je, že zařízení by nemělo mít OLED displej, aby nedošlo k takzvanému
\textit{burn-in} \citep[viz][]{OledBurnIn}. Druhý je, že zařízení by mělo být v produkčním provozu
bezpečně uzavřeno v krabičce, nebo by se mělo nacházet na bezpečném místě, aby se předešlo
nepovolené manipulaci.

\section{Autentifikace}

Zařízení se autentifikuje naskenováním QR kódu, jenž lze najít ve webové aplikaci. Ten obsahuje
JSON řetězec s aktivačním heslem, pomocí kterého se zařízení přihlásí do systému a získa svou konfiguraci.

\section{Cachování výsledků pro přesnost}

Výsledek z knihovny OpenALPR je seznam dvojic udávající SPZ a šanci, že konkrétní SPZ je správně --
čili jak si OpenALPR věří ve výsledek. Je tudíž logické měření udělat víc a provést aritmetický průměr a
zvolit nejlepší výsledek.

K ukládání takto dočasných dat (přibližně počet měření krát 1 sekunda) se databáze nehodí, a proto
bylo zavedeno ukládání do mezipaměti. V současné chvíli se využívá prostá paměť Backendu,
kde klíčem je $id$ zařízení. Díky tomu, že Node.js běží na jednom vlákně, nemusíme se bát souběhu
(angl. race-condition). Externí mezipaměť by bylo vhodné využít (např. Redis), pokud bychom spouštěli více
instancí Backend a prováděli takzvaný load-balancing.

Výchozí počet měření je 2, a lze ho upravit v nastavení Backendu.
