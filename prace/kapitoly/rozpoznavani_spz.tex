
\chapter{Rozpoznávání SPZ}

\noindent
Jak již bylo řečeno v kapitole \ref{archtech}, mobilní aplikace pořídí snímek,
pošle ho na Backend, jenž ho pošle serveru s knihovnou OpenALPR, která
rozpozná SPZ a výsledek pošle zpět na backend. V této kapitole si popíšeme
mobilní aplikaci a server s OpenALPR.

\section{Zvyšování přesnosti}

\subsection{Cachování výsledků}

\noindent
Výsledek z knihovny OpenALPR je seznam dvojic udávající SPZ a šanci, že konkrétní SPZ je správně --
jak si OpenALPR věří ve výsledek. Je tudíž logické měření udělat víc a provést aritmetický průměr a
zvolit nejlepší výsledek.

K ukládání takto dočasných dat (přibližně počet měření krát 1 sekunda) se databáze nehodí, a proto
bylo zavedeno ukládání do mezipaměti. V současné chvíli se využívá prostá paměť backendu,
kde klíčem je $id$ zařízení. Díky tomu, že Node.js běží na jednom vlákně, nemusíme se bát souběhu
(angl. race-condition). Externí mezipaměť by bylo vhodné využít (např. Redis), pokud by se spouštělo více
instancí backendu a prováděl by se takzvaný \textit{load-balancing}.

Výchozí počet měření je 2, a lze ho upravit v konfiguraci backendu.

\subsection{Filtrování podle geometrického obsahu}

\noindent
Pokud OpenALPR nalezne SPZ, udá i její pozici ve zdrojovém obrázku.
Aby se tedy předešlo naskenování SPZ, které jsou například daleko, lze odfiltrovat SPZ
podle jejich obsahu v pixelech čtverečních. Konkrétní hodnota je potřeba odladit na místě skenování a
lze změnit ve webové aplikaci pro kterékoliv zařízení.

\section{Autentifikace}

\noindent
Zařízení se autentifikuje naskenováním QR kódu, jenž lze najít ve webové aplikaci. Ten obsahuje
JSON řetězec s aktivačním heslem, pomocí kterého se zařízení přihlásí do systému a získa svou konfiguraci.

Samotné skenování QR kódu je provedeno externí aplikací Barcode Scanner od vývojáře
Zxing Team, která lze nainstalovat z Play Store.

Konkrétní mechnismus komunikace s touto externí byl převzán. \citep[viz][]{QrScan}

\section{Volba zařízení pro mobilní aplikaci}

\noindent
Co se týče hardwarového vybavení snímacího zařízení, tak je vyžadována přední kamera
s rozlišením alespoň 1000 na 1000 pixelů. Důvod pro toto rozlišení je, že backend obdržený snímek
stejně zmenší na 1000x1000 pixelů (lze však změnit ve webové aplikaci), aby knihovna OpenALPR provedla rozpoznání co nejrychleji, a zároveň
aby bylo rozpoznání dostatečně přesné. Procesor, RAM i vnitřní paměť může být libovolná -- kterékoliv
dnešní nové zařízení bohatě postačí (za předpokladu, že vnitřní paměť není zaplněná).
Minimální verze Androidu je 5 (SDK 21).

Autorovi se nepodařilo najít způsob, jak zároveň pořizovat v pravidelném intervalu snímky
a mít zařízení uzamknuté proti přístupu. K zajištění pořizování snímků si hlavní
obrazovka aplikace řekně systému Android o zabránění uzamknutí.
To má dva následky. První je, že zařízení by nemělo mít OLED displej, aby nedošlo k takzvanému
\textit{burn-in} \citep[viz][]{OledBurnIn}. Druhý je, že zařízení by mělo být v produkčním provozu
bezpečně uzavřeno v krabičce, nebo by se mělo nacházet na bezpečném místě, aby se předešlo
nepovolené manipulaci.

\section{Životní cyklus mobilní aplikace}

\noindent
Na obrázku \ref{fig:app_lifecycle} lze vidět životní cyklus mobilní aplikace.
Proces neprobíhá na jednom vlákně. Jakmile se pořídí fotografie, tak začně odpočet kolem jedné
sekundy, po kterém se pořídí další, a zároveň se už posílá první fotografie.
Změní-li se konfigurace na backendu, tak je poslána zařízení při dalším kontaktu, jinak
konfigurace poslána není.

\begin{figure}[!htb] \centering
  \includegraphics[width=135mm]{../img/app_lifecycle.jpg}
  \caption{Životní cycklus mobilní aplikace.}
  \label{fig:app_lifecycle}
\end{figure}

\section{Uživatelské rozhraní mobilní aplikace}

\noindent
Uživatelské rozhraní se skládá ze dvou obrazovek. Na obrázcích \ref{fig:app_ui}
lze vidět obě -- hlavní obrazovku a nastavení.

\begin{figure}[!htb] \centering
  \includegraphics[width=70mm]{../img/app_settings.png}
  \includegraphics[width=70mm]{../img/app_mainscreen.png}
  \caption{Rozhraní mobilní aplikace.}
  \label{fig:app_ui}
\end{figure}

V nastavení lze nastavit adresu backendu a vybrat si mezi HTTP a HTTPS.

\section{Implementační detaily mobilní aplikace}

\subsection{Komunikace s backendem}

\noindent
Ke komunikaci přes HTTP využívá aplikace knihovnu Volley, která je doporučena
v Android dokumentaci. Princip použití je takový, že si vytvoříme
\textit{singleton} obstarávající frontu žádostí, kterému předáváme HTTP žádosti s
\textit{callback} funkcí obsluhující odpověď. \citep[viz][]{Volley1}

\subsection{Ukládání snímků}

\noindent
Snímky se ukládají do paměti určené pro aplikaci. Kdyby se použili dočasné soubory,
mohlo by se stát, že je systém před posláním nemilosrdně smaže.
\citep[viz][]{AndroidMem}
Aplikace tedy musí obstarávat i mazání souborů, což se provádí
ihned po odeslání snímku na backend.
