%%% Hlavní soubor. Zde se definují základní parametry a odkazuje se na ostatní části. %%%

%% Verze pro jednostranný tisk:
% Okraje: levý 40mm, pravý 25mm, horní a dolní 25mm
% (ale pozor, LaTeX si sám přidává 1in)
\documentclass[12pt,a4paper]{report}
\setlength\textwidth{145mm}
\setlength\textheight{247mm}
\setlength\oddsidemargin{15mm}
\setlength\evensidemargin{15mm}
\setlength\topmargin{0mm}
\setlength\headsep{0mm}
\setlength\headheight{0mm}
% \openright zařídí, aby následující text začínal na pravé straně knihy
\let\openright=\clearpage

%% Pokud tiskneme oboustranně:
% \documentclass[12pt,a4paper,twoside,openright]{report}
% \setlength\textwidth{145mm}
% \setlength\textheight{247mm}
% \setlength\oddsidemargin{14.2mm}
% \setlength\evensidemargin{0mm}
% \setlength\topmargin{0mm}
% \setlength\headsep{0mm}
% \setlength\headheight{0mm}
% \let\openright=\cleardoublepage

%% Vytváříme PDF/A-2u
\usepackage[a-2u]{pdfx}
\usepackage{listings}
\def\inline{\lstinline[basicstyle=\ttfamily,keywordstyle={}]}

%% Přepneme na českou sazbu a fonty Latin Modern
\usepackage[czech]{babel}
\usepackage{lmodern}
\usepackage[T1]{fontenc}
\usepackage{textcomp}
\usepackage{multicol}
%% Použité kódování znaků: obvykle latin2, cp1250 nebo utf8:
\usepackage[utf8]{inputenc}

%%% Další užitečné balíčky (jsou součástí běžných distribucí LaTeXu)
\usepackage{amsmath}        % rozšíření pro sazbu matematiky
\usepackage{amsfonts}       % matematické fonty
\usepackage{amsthm}         % sazba vět, definic apod.
\usepackage{bbding}         % balíček s nejrůznějšími symboly
			    % (čtverečky, hvězdičky, tužtičky, nůžtičky, ...)
\usepackage{bm}             % tučné symboly (příkaz \bm)
\usepackage{graphicx}       % vkládání obrázků
\usepackage{fancyvrb}       % vylepšené prostředí pro strojové písmo
\usepackage{indentfirst}    % zavede odsazení 1. odstavce kapitoly
\usepackage{natbib}         % zajištuje možnost odkazovat na literaturu
			    % stylem AUTOR (ROK), resp. AUTOR [ČÍSLO]
\usepackage[nottoc]{tocbibind} % zajistí přidání seznamu literatury,
                            % obrázků a tabulek do obsahu
\usepackage{icomma}         % inteligetní čárka v matematickém módu
\usepackage{dcolumn}        % lepší zarovnání sloupců v tabulkách
\usepackage{booktabs}       % lepší vodorovné linky v tabulkách
\usepackage{paralist}       % lepší enumerate a itemize
\usepackage{listings}
\usepackage{fancyhdr}
\usepackage{pdfpages}
\usepackage{dirtree}
\usepackage{subfig}
% \usepackage{mathptmx}


\hyphenation{end-point}
\hyphenation{Graph-QL}

%%% Údaje o práci

\def\NazevSkoly{Gymnázium, Praha 6, Arabská 14}
% Název oboru včetně počátečního 'Obor'.
\def\NazevOboru{Obor programování}

% Název práce v jazyce práce (přesně podle zadání)
\def\NazevPrace{Parkovací systém}

% Název práce v angličtině
\def\NazevPraceEN{Parking System}

% Jména autorů
% Abecedně podle příjmení
\def\AutorPrace{Tomáš~Černý}

% Rok odevzdání
\def\RokOdevzdani{2020}
% Měsíc odevzdání
\def\MesicOdevzdani{Březen}

% Vedoucí práce: Jméno a příjmení s~tituly

% Nepovinné poděkování (vedoucímu práce, konzultantovi, tomu, kdo
% zapůjčil software, literaturu apod.)
\def\Podekovani{%
}

% Abstrakt (doporučený rozsah cca 80-200 slov; nejedná se o zadání práce)
\def\Abstrakt{%
Cílem projektu bylo navrhnout a implementovat informační systém pro správu
komerčních parkovišť, parkovišť supermarketů, obchodních center a podobných míst.
Systém je napsán moderními technologiemi a vyžaduje málo obsluhy za pomocí
automatického rozpoznávání SPZ díky knihovně OpenALPR a levným Android
zažízením.
Systém umožňuje vytvářet flexibilní pravidla určující cenu
parkování za jednotku času, jednotky času zadarmo a jejich interval platnosti.
Také jsou schopna filtrovat vozidla na základě rozpoznané SPZ.
Zároveň umožňuje monitorovat stav v reálném čase a číst statistiky.

Systém by po několika úpravách a přidání komunikace s
platebním terminálem a závorou měl být reálně použitelný.
}
\def\AbstraktEN{%
The goal of this project was to design and implement an information system for
administration of commercial parking lots, parking lots of supermarkets,
shopping centers and similar.
The system is written using modern technologies and requires little maintenance
by automatically recognizing license plates thanks to the OpenALPR library
and cheap Android devices.
It allows to create flexible rules that specify price per unit time,
free units of time. They can also be enabled during certain time periods and
can filter vehicles by their license plate.
The administrator can also watch both live and past statistics.

After the addition of a payment terminal and a parking barrier and some other changes,
the system should be usable in real-life.
}

% 3 až 5 klíčových slov (doporučeno), každé uzavřeno ve složených závorkách
\def\KlicovaSlova{%
{rozpoznávání SPZ}, {počítačové vidění}, {React}, {Typescript}, {GraphQL}
}
\def\KlicovaSlovaEN{%
{license plate recognition}, {computer vision}, {React}, {Typescript}, {GraphQL}
}

%% Balíček hyperref, kterým jdou vyrábět klikací odkazy v PDF,
%% ale hlavně ho používáme k uložení metadat do PDF (včetně obsahu).
%% Většinu nastavítek přednastaví balíček pdfx.
\hypersetup{unicode}
\hypersetup{breaklinks=true}

%% Definice různých užitečných maker (viz popis uvnitř souboru)
\include{makra}

%% Titulní strana a různé povinné informační strany
\begin{document}

\includepdf[pages={1-},width=\paperwidth]{soc_prefix.pdf}

\newpage
\pagenumbering{gobble}

%%% Povinná informační strana bakalářské práce

\openright

\vbox to 0.5\vsize{
\setlength\parindent{0mm}
\setlength\parskip{5mm}

Název práce:
\NazevPrace

Autoři:
\AutorPrace

% TODO: Je Lána vedoucí?
% Vedoucí práce:
% \Vedouci, \KatedraVedouciho

Abstrakt:
\Abstrakt

% Klíčová slova:
% \KlicovaSlova

\vss}\nobreak\vbox to 0.49\vsize{
\setlength\parindent{0mm}
\setlength\parskip{5mm}

% Opakování v angličtině.

Title:
\NazevPraceEN

Authors:
\AutorPrace

Abstract:
\AbstraktEN

\vss}

\newpage

\openright
\pagestyle{plain}
\pagenumbering{arabic}
\setcounter{page}{1}


%%% Strana s automaticky generovaným obsahem bakalářské práce

\tableofcontents

%%% Jednotlivé kapitoly práce jsou pro přehlednost uloženy v samostatných souborech
\chapter{Úvod} \label{uvod}

Tato dokumentace se zabývá koncepcí a implementací informačního systému pro
parkoviště.

\section{Požadavky na softwarové řešení}

<textem popsat schopnosti>
% cheap
% automated
% 

\section{Cíle práce}

\begin{itemize}
  \item Naskenování SPZ
  \item Vytváření pravidel - dostatečně flexibilní pro většinu použití (asi obejdu pár parkovišť a udělám generické řešení pravidel)
    \begin{itemize}
      \item Nastavitelné situace upozornění
      \item od Ah do Bh tarif X, Zh zadarmo v den D, svátky
      \item limity na hodiny zdarma
    \end{itemize}
  \item Statistiky (grafy etc.) počtu aut a dalších relevantních dat
  \end{itemize}


\chapter{Architektura a technologie}

\section{Architektura řešení} \label{architektura_reseni}

\begin{figure} \centering
  \includegraphics[width=145mm]{../img/architecture_drawing.jpg}
  \caption{Diagram komponent a jejich komunikace.}
  \label{fig:architecture_drawing}
\end{figure}

Parkovací systém se zkládá z následujících částí, které si nyní popíšeme stručně a detailněji později.
Části spolu komunikují pomocí HTTP.
Obrázek \ref{fig:architecture_drawing} ukazuje tyto části a nastiňuje způsob komunikace.

\begin{itemize}
  \item \textbf{Backend} je středobodem celé aplikace -- komunikuje se všemy ostatními komponentami.
        Zajišťuje business logiku aplikace, autentizaci i autorizaci uživatelů a perzistenci dat do \textbf{Databáze}.
  \begin{itemize}
    \item \textbf{Databáze} slouží k ukládání a čtení dat.
    \item \textbf{OpenALPR Server} (převzato z https://github.com/gerhardsletten/express-openalpr-server) je server, jenž obstarává přístup
          ke knihovně OpenALPR (https://github.com/openalpr/openalpr), která rozpoznává SPZ, přes protokol http.
  \end{itemize}
  \item \textbf{Mobilní aplikace} posílá obrazová data na \textbf{Backend}, kde jsou zpracována. Je určena pro platformu Android.
  \item \textbf{Frontend} je rozhraní mezi celým systémem a správcem parkoviště a dalšího personálu.
\end{itemize}

\section{Technologie}

\subsection{Databáze}

Databáze MongoDB, která byla vybrána, protože data se budou převážně zapisovat a bude potřeba v nich rychle hledat a provádět agregační dotazy.
Mimo jiné umožňuje provoz několika spolupracujících instancí, zálohování apod \citep[viz][]{MongoDB}.

\subsection{Backend} \label{backend}

Jako programovací jazyk pro \textbf{Backend} byl zvolen staticky typovaný Typescript kvůli rychlosti vývoje
a množství knihoven, které poskytuje ekosystém Node.js.

Pro definici databázových modelů a komunikaci s MongoDB byla kvůli své vyspělosti a skvělé funkcionalitě zvolena
knihovna mongoose.

Primárním způsobem komunikace s \textbf{Frontend}
je dotazovací jazyk GraphQL, který přináší ucelený popis poskytovaných dat pomocí kontroly typů,
expresivních dotazů, jejichž odpověď má stejný "tvar" v JSON formátu.
Obrázek \ref{fig:graphql_example} ukazuje dotaz hledání uživatele podle jména.

\begin{figure} \centering
\includegraphics[width=145mm]{../img/graphql_example.png}
\caption{Příklad GraphQL dotazu (vlevo) a odpovědi (vpravo). Screenshot z nástroje GraphQL Playground.}
\label{fig:graphql_example}
\end{figure}

Model uživatele může mít i další atributy, ale GraphQL vrátí přesně ty údaje, na které se uživatel zeptal.
Tento triviální příklad neukazuje další funkce jako mutace dat, dědičnost typů, více dotazů v jednom http dotazu
a mnoho dalších.
GraphQL je pouze specifikace vytvořená společností Facebook a má několik
implementací. Pro tento projekt byla zvolena implementace Apollo \citep[viz][]{Apollo}. Detailní informace jsou dostupné na oficiálních
stránkach https://graphql.org/.

Jelikož GraphQL posílá odpovědi v JSON, není vhodné na posílání obrázků. Je to možné za využití base64 kódování,
ale přes síť se přenese více bytů, než při použití obvyklého způsobu přes HTTP. Z toho důvodu pro posílání
obrázků (např. QR kódu pro Frontend pro autentifikaci zařízení nebo zaznamenané znímky SPZ) bude mít Backend i
klasické REST endpointy.

\subsection{Frontend} \label{frontend}

Pro \textbf{Frontend} byl jako u \textbf{Backend} vybrán Typescript ze stejných důvodů. Webové rozhraní
je takzvaná SPA (z angl. Single-Page-Application), což znamená, že uživateli se obsah mění dynamicky
bez načítání dalších stránek.
Renderování zajišťuje knihovna React, která od klasického přístupu, kdy se odděluje HTML a Javascript do separátních
souborů, mandatuje, že v jednom souboru je jedna komponenta se vším svým HTML a logikou ve formě Javascriptu nebo
Typescriptu. Pomocí další knihovny typestyle pak můžeme do stejného souboru psát i typované CSS.
Aby bylo možné snadno sdílet mezi komponentami stav, byla pro takzvaný state-management zvolena knihovna Redux.
Diagram na obrázku \ref{fig:react_redux_dataflow} ukazuje tok dat mezi Reactem a Reduxem.

Jelikož je aplikace vyvíjena pro správce systému a ne pro velké množství uživatelů, můžeme si dovolit
klást menší nároky na velikost aplikace (tj. můžeme přidávat i velké knihovny), což velice usnadní vývoj za
nízkou cenu.

\begin{figure}[!htb] \centering
\includegraphics[width=145mm]{../img/react-redux-architecture.png}
\caption{Spojení React a Redux. \citep[viz][]{react_redux_dataflow}}
\label{fig:react_redux_dataflow}
\end{figure}

\subsection{Mobilní aplikce} \label{mobile_app}

\textbf{Mobilní aplikace}, která je určena pro platformu Android, měla volbu jazyka omezenou na Javu, Kotlin a C++.
Rychlost C++ není potřeba a navíc autor s tímto nízkoúrovňovým jazykem nemá takové zkušenosti.
Kotlin oproti Javě umožňuje přímočarejší přístup k prkům uživatelského rozhraní, a proto byl zvolen.

Jediným úkolem \textbf{Mobilní aplikace} je v pravidelném intervalu pořídit snímek fotoaparátem a poslat ho na
\textbf{Backend}, který jej patřičně zpracuje.

\subsection{Detekce SPZ}

Detekci SPZ bude zajišťovat knihovna OpenALPR \citep[viz][]{OpenALPR}, jejímž vstupem je obrázek a popřípadě
parametry jako úhel kamery apod. Ke zbytku aplikace bude připojena malým HTTP serverem, jenž byl převzán a upraven
\citep[viz][]{OpenALPR_Server}.
Ten umožňuje poslat pomocí protokolu HTTP obrázek a obdržet JSON s SPZ daty a souřadnicemi detekované SPZ.
Samotný server ke knihovně přistupuje zavoláním binárky \textit{alpr}, který jako argument přijme cestu k
obrázku, ve kterém hledáme SPZ. Alternativní a lepší způsob přístupu by bylo mít v Node.js přímo takzv.
\textit{language-binding}, ale to se autorovi (a mnoho dalším, kteří se o to pokoušeli) nepodařilo.

\section{Metodika vývoje}

Nejprve se vyvine veškerá funkcionalita v základní podobě
(na způsob MVP -- z angl. Minimal-Viable-Product) a později se vše vyhladí a zlepší. To však neznamená,
že bychom si nedali záležet na kvalitním a udržitelném kódu, naopak. Cílem je mít flexibilní základ,
na kterém lze stavět. Pro klidný spánek budeme psát dle uvážení automatizované testy, abychom předešli
nežádoucímu chování aplikace (takzvaná regrese) po úpravě kódu.

Backend i frontend budou vyvinuty současně. Dokud není mobilní aplikace pro zařízení, lze ji simulovat
například nástrojem curl.


\chapter{Webová aplikace}

\section{Obrazovky}

Dle požadavků z úvodu mějme po příhlášení do webového rozhraní následující obrazovky, mezi kterými bude uživatel přepínat pomocí hlavního menu.

\begin{itemize}
  \item \textbf{Dashboard} -- jednoduché shrnutí nedávných statistik, kolik vozidel je
        momentálně na parkovišti, aktivní zařízení apod.
  \item \textbf{Statistiky} -- detailněji zobrazené údaje o počtu parkování a výdělku podle roku, měsíce a dne s grafy.
        Zde půjde i exportovat data to CSV souboru (\textit{Comma-Separated-Values}).
  \item \textbf{Pravidla a Filtry} -- definice parkovacích pravidel a filtrů vozidel (popsáno v \ref{analysis_parking_schema}).
        Pro ověření půjde si parkovací pravidla a filtry odsimulovat.
  \item \textbf{Zařízení} -- správa zařízení zachycujících fotografie SPZ, která lze autentifikovat do systému pomocí
        QR kódu.
  \item \textbf{Správa uživatelů} -- přidávání, odebírání a úprava uživatelů a jejich operávnění.
  \item \textbf{Správa účtu} -- změna údajů a hesla současně přihlášeného uživatele.
\end{itemize}

\section{Autentizace a autorizace} \label{auther_authen}

Autentifikace lidských uživatelů bude probíhat pomocí standardního hesla a zařízení pomocí dlouhého, náhodně generovaného hesla,
které bude posíláno na Frontend ve formě QR kódu, které zařízení naskenuje.
TODO: Tokens

Autorizace pak bude probíhat bez Cookies, místo toho bude Access Token poslán v hlavičce HTTP hlavičce
\textit{Authorization}, čímž předejdeme CSRF/XSRF útokům.

\section{Business logika}

\subsection{Parkovací pravidla} \label{analysis_parking_schema}

\begin{itemize}
  \item Různá vozidla mohou podléhat ruzným pravidlům.
  \item Pravidla mají prioritu.
  \item Pravidla mají časové omezení.
  \item V jednu chvíli může platit více pravidel.
\end{itemize}

Mechanismus, kterým umožníme vozidlům být ovlivněna některými pravidly,
budou filtry.
Pro dostatečnou flexibilitu je zapotřebí oddělit samotná pravidla od jejich
priority, časového intervalu platnosti i filtrů,
k čemuž bude sloužit objekt typu \texttt{ParkingRuleAssignment}.

% TODO - add a db schema image that takes less space
\begin{lstlisting}
type ParkingRuleAssignment {
  rules: [ParkingRule]!
  start: DateTime!
  end: DateTime!
  # ALL nebo NONE
  vehicleFilterMode: VehicleFilterMode!
  vehicleFilters: [VehicleFilter!]
  priority: NonNegativeInt!
}

type VehicleFilter {
  id: ID!
  name: String!
  action: VehicleFilterAction!
  vehicles: [Vehicle!]!
}
\end{lstlisting}

Filtrování bude mít dva módy: začneme se všemi vozidly (ALL) a začneme bez vozidel (NONE).
Následné filtry mohou buď přidávat, nebo odstraňovat jednotlivá vozidla.
Hodí se mít filtry uložené separátně, aby mohli být využity několikrát.

Pro zjednodušení algoritmů, uvalíme omezení: ve stejný čas nesmí existovat více \texttt{ParkingRuleAssignment}
se stejnou prioritou.

\subsubsection*{Algoritmus filtru vozidel}

\textit{Vstup: objekt \texttt{ParkingRuleAssignment}, vozidlo}

\textit{Výstup: boolean určující platnost}
\begin{enumerate}
  \item Na základě módu filtrování si budeme udržovat množinu buď odstraněných vozidel (mód ALL), nebo přidaných vozidel (mód NONE).
  \item Podle příslušné akce filtrů (odstranit nebo přidat) budeme množinu našich vozidel manipulovat. Např. je-li mód ALL a filtr odstraňuje, do množiny si vozidla přidáme.
  \item Pokud je vozidlo ve výsledné množině, tak pro něj \texttt{ParkingRuleAssignment} platí pokud je filtrovací mód NONE a neplatí pokud je mód ALL. Opačné výsledky nastanou, pokud vozidlo v množině není.
\end{enumerate}

Časová i paměťová složitost algoritmu je ${\cal O}(N)$, kde $N$ je počet filtrů.

Tento algoritmus lze potenciálně rozdělit na předvýpočet (kroky 1. a 2.) a ověření (krok 3.).
To je jedna možnost, která zvýší paměťovou náročnost, protože bychom si museli pamatovat množiny, což je nepraktické.
Je-li uživatel příčetný, počet použitých filtrů nebude obrovský, a tudíž je lepší zvolit následující způsob cachovaní.
Zapamatujeme si výsledky pro určitý seznam filtrů pro konkrétní vozidlo pouze při běhu algoritmu, který je
vysvětlen v následující kapitole, čímž následující algoritmus zrychlíme.

\subsubsection*{Algoritmus pro aplikaci ParkingRuleAssignmentů}

\begin{figure}[!htb] \centering
  \includegraphics[width=145mm]{../img/rules_drawing.jpg}
  \textit{Modré úsečky neplatí kvůli filtrům nebo protože jsou deaktivované. Oranžová čára značí výstup požadovaného algoritmu.}
  \caption{Ilustrace problému úseček.}
  \label{fig:rules_drawing}
\end{figure}

V jednom čase může existovat více objektů typu \texttt{ParkingRuleAssignment} avšak s různou prioritou.
Může se stát, že aplikovaných \texttt{ParkingRuleAssignment} bude několik (různé priority, vyprší platnost, etc.).

Situaci si lze představit jako několik úseček navzájem rovnoběžných úseček v různých výškách, které se neprotínají.
Nás nyní zajímá, na které a v jakých intervalech na ně dopadne světlo, pokud na ně kolmo zeshora posvítíme.
Situaci lze vidět na obrázku \ref{fig:rules_drawing}.

Pro zjednodušení předpokládejme, že všechny \texttt{ParkingRuleAssignment}, které zpracováváme, platí pro naše vozidlo.
Přidat tuto kontrolu později je triviální.

\textit{Vstup: seznam \texttt{ParkingRuleAssignmentů} odpovídající pro interval pobytu vozidla na parkovišti, vozidlo}

\textit{Výstup: seznam \texttt{ParkingRuleAssignmentů} s časy platnosti}
\begin{enumerate}
  \item Seřadíme si začátky a konce úseček podle jejich času.
  \item Vytvoříme si haldu pro odkládání úseček, která řadí podle priority -- větší výše.
  \item Vytvoříme si seznam aplikovaných pravidel s časy (časy mohou se lišit od počátečních i koncových časů).
  \item Nechť \textit{s} je současná úsečka a \textit{$t_s$} čas zvolení \textit{s} (čas zvolení se může lišit od začátku úsečky).
  \item Pro každou událost \textit{u} značící začátek/konec úsečky (aplikaci pravidla) \textit{p}:
  \begin{enumerate}
    \item Pokud se jedná o začátek nové úsečky:
    \begin{enumerate}
      \item Pokud není zvolená úsečka:\\
            \textit{$t_s$} $\leftarrow$ \textit{p.start}\\
            \textit{s} $\leftarrow$ \textit{p}
      \item Pokud je zvolená úsečka a \textit{p} má vyšší prioritu než \textit{s}:\\
            \textit{s} dáme do seznamu aplikovaných pravidel se začátkem \textit{$t_s$} a koncem \textit{p.start}.\\
            \textit{s} dáme na haldu, pokud \textit{s.end} > \textit{p.end}.\\
            \textit{$t_s$} $\leftarrow$ \textit{p.start}\\
            \textit{s} $\leftarrow$ \textit{p}
      \item Pokud je zvolená úsečka a \textit{p} má nižší prioritu než \textit{s} a \textit{p.end} > \textit{s.end}:\\
            \textit{s} dáme na haldu
    \end{enumerate}
    \item Jinak (jedná se o konec nějaké úsečky):
    \begin{enumerate}
      \item Přidáme \textit{s} do seznamu aplikovaných pravidel se začátkem \textit{$t_s$} a koncem \textit{s.end}.
      \item Taháme z haldy, dokud nedostaneme úsečku s koncem později než konecm \textit{p}, nebo dokud halda není prázdná.
      \item Pokud jsme z haldy vhodnou úsečku vytáhli, použijeme ji. V opačném případě vyprázdníme \textit{s} a \textit{$t_s$}.
    \end{enumerate}
  \end{enumerate}
\end{enumerate}

Algoritmus zajisté doběhne, protože máme konečný počet událostí a v každém cyklu jednu zpracujeme.
Při rozumném počtu \texttt{ParkingRuleAssignment} v daném intervalu je algoritmus velice rychlý.
${\cal O}(N^2\cdot logN)$
${\cal O}(N\cdot logN)$
Přidáme-li filtrování, které nemusíme provést pro všechny úsečky, ${\cal O}(N\cdot M)$.

\section{GraphQL resolvery}

<TODO: EXPLAIN RESOLVERS>

Jelikož je většina operací nad modeli se opakuje, byli napsány obecné resolvery jako HOF (Higher-Order-Function)
pro vytváření, úpravu, mazání, vyhledávání a získávání modelů v relaci.
Měnící se část je pak pouze definice databázového modelu.

\section{Databázový model}

\section{Uživatelské rozhraní}

\subsection{Obecná vybírátka modelů}

Dle principu neopakování se bylo vytvořeno několik obecných UI komponent, které umožňují vyhledávání libovolných
modelů a jejich volbu a využití v ostatních komponentách. Ve zdrojovém kódu je implementujeme jako
HOF (Higher-Order-Function), což je funkce vracející další funkce -- konkrétní komponenty.
Měnící se části, které do těchto obecných komponent budeme vkládat je komponenta renderující jediný model
(\textit{renderModel}),
GraphQL dotaz pro získávání modelů (\textit{queryString}), funkce, jejímž vstupem je odpověď na GraphQL dotaz a výstupem je
samotné pole modelů (\textit{modelArrayGetter}), a funkce, jejímž vstupem je dotazovací řetězec a výstupem jsou
argumenty pro GraphQL dotaz (\textit{identifierToOptions}).

Obrázek \ref{fig:picker_lifecycle} ukazuje tok dat v jednom z obecných vybírátek.
Obrázek \ref{fig:picker_component} ukazuje vyrenderovanou komponentu téhož vybírátka.

\begin{figure}[!htb] \centering
  \includegraphics[width=145mm]{../img/picker_lifecycle.jpg}
  \caption{Tok dat v jednom z obecných vybírátek.}
  \label{fig:picker_lifecycle}
\end{figure}

\begin{figure}[!htb] \centering
  \includegraphics[width=70mm]{../img/picker_component.png}
  \caption{Vyrenderovaná komponenta jednoho vybírátka.}
  \label{fig:picker_component}
\end{figure}


\chapter{Rozpoznávání SPZ}

\noindent
Jak již bylo řečeno v kapitole \ref{archtech}, mobilní aplikace pořídí snímek,
pošle ho na Backend, jenž ho pošle serveru s knihovnou OpenALPR, která
rozpozná SPZ a výsledek pošle zpět na backend. V této kapitole si popíšeme
mobilní aplikaci a server s OpenALPR.

\section{Zvyšování přesnosti}

\subsection{Cachování výsledků}

\noindent
Výsledek z knihovny OpenALPR je seznam dvojic udávající SPZ a šanci, že konkrétní SPZ je správně --
jak si OpenALPR věří ve výsledek. Je tudíž logické měření udělat víc a provést aritmetický průměr a
zvolit nejlepší výsledek.

K ukládání takto dočasných dat (přibližně počet měření krát 1 sekunda) se databáze nehodí, a proto
bylo zavedeno ukládání do mezipaměti. V současné chvíli se využívá prostá paměť backendu,
kde klíčem je $id$ zařízení. Díky tomu, že Node.js běží na jednom vlákně, nemusíme se bát souběhu
(angl. race-condition). Externí mezipaměť by bylo vhodné využít (např. Redis), pokud by se spouštělo více
instancí backendu a prováděl by se takzvaný \textit{load-balancing}.

Výchozí počet měření je 2, a lze ho upravit v konfiguraci backendu.

\subsection{Filtrování podle geometrického obsahu}

\noindent
Pokud OpenALPR nalezne SPZ, udá i její pozici ve zdrojovém obrázku.
Aby se tedy předešlo naskenování SPZ, které jsou například daleko, lze odfiltrovat SPZ
podle jejich obsahu v pixelech čtverečních. Konkrétní hodnota je potřeba odladit na místě skenování a
lze změnit ve webové aplikaci pro kterékoliv zařízení.


\section{Autentifikace}

\noindent
Zařízení se autentifikuje naskenováním QR kódu, jenž lze najít ve webové aplikaci. Ten obsahuje
JSON řetězec s aktivačním heslem, pomocí kterého se zařízení přihlásí do systému a získa svou konfiguraci.

Samotné skenování QR kódu je provedeno externí aplikací Barcode Scanner od vývojáře
Zxing Team, která lze nainstalovat z Play Store.

Konkrétní mechnismus komunikace s touto externí byl převzán. \citep[viz][]{QrScan}

\section{Volba zařízení pro mobilní aplikaci}

\noindent
Co se týče hardwarového vybavení snímacího zařízení, tak je vyžadována přední kamera
s rozlišením alespoň 1000 na 1000 pixelů. Bližší informace ohledně a zdůvodnění zmenšení jsou v sekci \ref{app_resizing}.
Procesor, RAM i vnitřní paměť může být libovolná -- kterékoliv
dnešní nové zařízení bohatě postačí (za předpokladu, že vnitřní paměť není zaplněná).
Minimální verze Androidu je 5 (SDK 21).

Autorovi se nepodařilo najít způsob, jak zároveň pořizovat v pravidelném intervalu snímky
a mít zařízení uzamknuté proti přístupu. K zajištění pořizování snímků si hlavní
obrazovka aplikace řekně systému Android o zabránění uzamknutí.
To má dva následky. První je, že zařízení by nemělo mít OLED displej, aby nedošlo k takzvanému
\textit{burn-in} \citep[viz][]{OledBurnIn}. Druhý je, že zařízení by mělo být v produkčním provozu
bezpečně uzavřeno v krabičce, nebo by se mělo nacházet na bezpečném místě, aby se předešlo
nepovolené manipulaci.

\section{Životní cyklus mobilní aplikace}

\noindent
Na obrázku \ref{fig:app_lifecycle} lze vidět životní cyklus mobilní aplikace.
Proces neprobíhá na jednom vlákně. Jakmile se pořídí fotografie, tak začně odpočet kolem jedné
sekundy, po kterém se pořídí další, a zároveň se už posílá první fotografie.
Změní-li se konfigurace na backendu, tak je poslána zařízení při dalším kontaktu, jinak
konfigurace poslána není.

\begin{figure}[!htb] \centering
  \includegraphics[width=135mm]{../img/app_lifecycle.jpg}
  \caption{Životní cycklus mobilní aplikace.}
  \label{fig:app_lifecycle}
\end{figure}

\section{Optimalizace velikosti přenesených dat} \label{app_resizing}

\noindent
Aby se ušetřilo na přenesených datech a aby rozpoznání proběhlo rychleji,
mobilní aplikace zmenší snímek
na nastavitelnou hodnotu
Tuto funkci poskytuje metoda \textit{createScaledBitmap} v třídě \textit{android.graphics.Bitmap}.

V závislosti na poměru stran snímků největšího rozlišení fotoaparátu,
může být potřeba hodnotu zmenšení změnit.
Základní hodnota je 1300x1000 pixelů, protože aplikace byla testována na
zařízení s fotoaparátem o roslišení 4356x3492 pixelů (poměr stran je 4:3).

Ušetření je opravdu veliké. Při testu s fotoaparátem o rozlišení 4356x3492 pixelů
byla průměrná velikost nezmenšených snímků $1,9$MB (JPG, $N=50$, $\sigma=0,325$).
Průměrná velikost snímků zmenšených na 1300x1000 pixelů byla
$0,15$MB (JPG, $N=50$, $\sigma=0,024$). Kdyby byla frekvence snímání jeden snímek za sekundu,
tak by bez optimalizací byl objem přenesených dat za jeden den $164,16$GB ($3600\cdot24\cdot1,9$MB).
Při stejné frekvenci s optimalizacemi by byl objem přenesených dat za jeden den
$12,96$GB ($3600\cdot24\cdot0,15$MB), což je $7,9$\% objemu bez optimalizací.

\section{Uživatelské rozhraní mobilní aplikace}

\noindent
Uživatelské rozhraní se skládá ze dvou obrazovek. Na obrázcích \ref{fig:app_ui}
lze vidět obě -- hlavní obrazovku a nastavení.

\begin{figure}[!htb] \centering
  \includegraphics[width=70mm]{../img/app_settings.png}
  \includegraphics[width=70mm]{../img/app_mainscreen.png}
  \caption{Rozhraní mobilní aplikace.}
  \label{fig:app_ui}
\end{figure}

V nastavení lze nastavit adresu backendu a vybrat si mezi HTTP a HTTPS.

\section{Implementační detaily mobilní aplikace}

\subsection{Komunikace s backendem}

\noindent
Ke komunikaci přes HTTP využívá aplikace knihovnu Volley, která je doporučena
v Android dokumentaci. Princip použití je takový, že si vytvoříme
\textit{singleton} obstarávající frontu žádostí, kterému předáváme HTTP žádosti s
\textit{callback} funkcí obsluhující odpověď. \citep[viz][]{Volley1}

\subsection{Ukládání snímků}

\noindent
Snímky se ukládají do paměti určené pro aplikaci. Kdyby se použily dočasné soubory,
mohlo by se stát, že je systém před posláním nemilosrdně smaže.
\citep[viz][]{AndroidMem}
Aplikace tedy musí obstarávat i mazání souborů, což se provádí
ihned po odeslání snímku na backend.

% 
\chapter{Uživalteská dokumentace} \label{uzivatelska_dokumentace}


\chapter{Instalace}

\noindent
Defaultně potřebuje aplikace porty 4500 (server rozpoznávající SPZ), 8080 (backend) a
8889 (frontend). Pokud je MongoDB instalována systémově, běží na portu 27017, a pokud
v Dockeru, tak běží na portu 5432.
Porty lze změnit v konfiguračních souborech jednotlivých
komponent. Při změně portu komponenty A, na které jsou jiné komponenty B závislé,
je potřeba změnit port závislých komponent B pro danou komponentu A.
Mobilní aplikace se kompiluje vývojovým prostředím Android Studio.

\section{Soubory nastavení}

\noindent
Soubory pro nastavení jsou:

\begin{itemize}
  \setlength\itemsep{-.3em}
  \item Backend
  \begin{itemize}
    \item /backend/Dockerfile
    \item /backend/config/production.json
    \item /backend/config/development.json
    \item /backend/src/config.ts
  \end{itemize}
  \item Frontend
  \begin{itemize}
    \item /frontend/Dockerfile
    \item /frontend/config/main.js
    \item /frontend/config/main.local.js
  \end{itemize}
  \item Server rozpoznávající SPZ
  \begin{itemize}
    \item /express-openalpr-server/Dockerfile
    \item /express-openalpr-server/processes.json
  \end{itemize}
\end{itemize}

\newpage
\section{Instalace Dockerem}

\noindent
Instalace celého systému Dockerem je velice jednoduchá, ten zajistí kompilaci,
spuštění, propojení všech částí a odhalení potřebných služeb ven na internet. \citep[][]{DockerDocs}

% Kromě Backendu, Frontendu, OpenALPR Server a MongoDB používá tato distribuce
% Dockerem populární HTTP server NGINX pro SSL a reverse proxy. Vnitřní HTTP komunikace
% tak nemusí být zabezpečena.

\subsubsection*{Postup instalace}

\begin{enumerate}
  \setlength\itemsep{.05em}
  \item Po stažení git repozitáře se zdrojovým kódem nastavíme submoduly (lze vynechat, pokud máte kód jako zip archiv):\\
  \begin{lstlisting}[numbers=none]
    $ git submodule init
    $ git submodule update --remote --recursive
  \end{lstlisting}
  \item Dle potřeby upravíme soubor \textit{/docker-compose.yml} a ostatní konfigurační soubory.
  \item Celý systém spustíme.\\
  \begin{lstlisting}[numbers=none]
    $ docker-compose up
  \end{lstlisting}
  \item Zjistíme počáteční přihlašovací údaje z výstupu předchozího příkazu. Pokud byla proměnná prostředí \texttt{NODE\_ENV=development},
  bude počátečení heslo 1234.\\\\
  \includegraphics[width=100mm]{../img/installation_pass.png}\\
  \item Otevřeme prohlížeč na adrese \url{http://localhost:8889}.
\end{enumerate}

Krok číslo 3 může trvat i několik minut v závislosti na rychlosti internetového
přípojení. Protože Docker zabalí vše včetně systémových závislostí, velikost
výsledných imagů je kolem $1,4$GB.

Pro detaily je potřeba konzultovat soubor \textit{/docker-compose.yml}, jednotlivé soubory
zvané \textit{Dockerfile} každé komponenty a konfigurační soubory komponent.

\section{Instalace bez Dockeru}

\noindent
Pro spuštění bez Dockeru je potřeba nainstalovat několik programů. Jmenovitě se
jedná o Typescript pro kompilaci, Node.js a NPM pro knihovny a spuštění backendu, frontendu
a serveru rozpoznávající SPZ, pm2 pro load-balancing serveru rozpoznávající SPZ, MongoDB,
Android Studio pro kompilaci mobilní aplikace a pro rozpoznávání SPZ je potřeba OpenALPR .

Instalace knihovny OpenALPR je nejproblematičtější, protože
probíhá kompilací velkého množství C++. Instrukce ke kompilaci
jsou dostupné z \url{https://github.com/openalpr/openalpr#compiling}. Další alternativou je spouštet server rozpoznávající SPZ
Dockerem a zbytek bez Dockeru -- stačí ve složce \textit{/express-openalpr-server} spustit příkazy: \textbf{asddddddddddddddddddddddddddddd}

\begin{lstlisting}[numbers=none]
  $ npm install
  $ docker build -t express-openalpr:latest "."
  $ docker run express-openalpr:latest 
\end{lstlisting}

\subsubsection*{Postup spuštění bez Dockeru}

\begin{enumerate}
  \setlength\itemsep{.05em}
  \item Nainstalujeme závislosti.
  \item Po stažení git repozitáře se zdrojovým kódem nastavíme submoduly (lze vynechat, pokud máte kód jako zip archiv):\\
  \begin{lstlisting}[numbers=none]
    $ git submodule init
    $ git submodule update --remote --recursive
  \end{lstlisting}
  \item Spustíme MongoDB.
  \item Spustíme server rozpoznávající SPZ.\\
  \begin{lstlisting}[numbers=none]
    $ cd express-openalpr-server
    $ npm install
    $ pm2 start processes.json
  \end{lstlisting}
  \item Spustíme backend buď následujícími příkazy, nebo bash skriptem \textit{/backend/sdev.sh -c}.\\
  \begin{lstlisting}[numbers=none]
    $ cd backend
    $ npm install
    $ npm run compile
    $ npm start
  \end{lstlisting}
  \item Spustíme frontend (poslední příkaz může chvíli trvat).\\
  \begin{lstlisting}[numbers=none]
    $ cd frontend
    $ npm install
    $ npm start
  \end{lstlisting}
  \item Přihlašovací údaje jsou \texttt{admin:1234}, protože nespouštíme s proměnnou prostředí \texttt{NODE\_ENV=production} jako při spuštění Dockerem.
  \item Otevřeme prohlížeč na adrese \url{http://localhost:8889}.
\end{enumerate}


\chapter*{Závěr}
\addcontentsline{toc}{chapter}{Závěr}

Výsledný projekt splňuje celé zádání kromě jednoho bodu -- konkrétně se jedná o
různý provoz
o svátích a podobných dnech (viz \ref{missing1}). Implementace takovéto funkcionality
vyžaduje implementovat kalendář.
Co se týče rozpoznávání SPZ, statistik, samotných pravidel a filtrů vozidel,
tak zde je vše implementováno a funguje skvěle.
Při prohlížení záznamů parkování lze nahlédnout na výřez SPZ z pořízeného snímku.
Navíc lze pravidla a filtry simulovat pro libovolné vozidlo.

Vývoj byl díky vhodně zvoleným technologiím poměrně rychlý a bezbolestný.
V jeho průběhu nedošlo k žádnému backtrackování kvůli předchozím rozhodnutím.
Projekt je bez velkých obtíží rozšiřitelný a pozměnitelný.
Dalším rozšířením, aby projekt byl plnohodnotný parkovací systém, by byla
integrace s platebním terminálem a závorou.


%%% Seznam použité literatury
\include{literatura}

%%% Obrázky v bakalářské práci
%%% (pokud jich je malé množství, obvykle není třeba seznam uvádět)
\listoffigures

%%% Tabulky v bakalářské práci (opět nemusí být nutné uvádět)
%%% U matematických prací může být lepší přemístit seznam tabulek na začátek práce.
%\listoftables

%%% Použité zkratky v bakalářské práci (opět nemusí být nutné uvádět)
%%% U matematických prací může být lepší přemístit seznam zkratek na začátek práce.
% \chapwithtoc{Seznam použitých zkratek}

%%% Přílohy k bakalářské práci, existují-li. Každá příloha musí být alespoň jednou
%%% odkazována z vlastního textu práce. Přílohy se číslují.
%%%
%%% Do tištěné verze se spíše hodí přílohy, které lze číst a prohlížet (dodatečné
%%% tabulky a grafy, různé textové doplňky, ukázky výstupů z počítačových programů,
%%% apod.). Do elektronické verze se hodí přílohy, které budou spíše používány
%%% v elektronické podobě než čteny (zdrojové kódy programů, datové soubory,
%%% interaktivní grafy apod.). Elektronické přílohy se nahrávají do SISu a lze
%%% je také do práce vložit na CD/DVD. Povolené formáty souborů specifikuje
%%% opatření rektora č. 72/2017.
% \appendix
% \chapter{Přílohy}

% \openright
\end{document}
